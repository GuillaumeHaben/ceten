%%%%%%%%%%%%%%%%%%%%%%%%%%%%%%%%%%%%%%%%%%%%%%%%%%%%%%%%%%%%%%%%%%%%%%%
%                                                                     %
%                        Statuts du Ceten                             %
%                                                                Beta %
%%%%%%%%%%%%%%%%%%%%%%%%%%%%%%%%%%%%%%%%%%%%%%%%%%%%%%%%%%%%%%%%%%%%%%%
%
%  Auteurs :
%    Philippe Becker <philippe.becker@telecomnancy.net
%    Kilian Cuny <kilian.cuny@telecomnancy.net
%    Julien Déoux <julien.deoux@telecomnancy.net>
%    Yoni Lévy <yoni.levy@telecomnancy.net
%    
%  Licence :
%    CC-BY-NC 4.0 (http://creativecommons.org/licenses/by-nc/4.0)
%
%%%



%%%%%%%%%%%%%%%%%%%
%  Configuration  %
%%%%%%%%%%%%%%%%%%%

\documentclass{article} % Jusqu'à preuve du contraire, les documents édités par le
                                                     % Ceten ne font pas 300 pages
\usepackage[a4paper,includeheadfoot,margin=2.54cm]{geometry}
\usepackage[francais]{babel}

\usepackage{hyperref}
\usepackage{graphicx} 
\usepackage{titlesec} 
\usepackage{ifxetex} 
\usepackage[usenames]{xcolor} 
\definecolor{newCeten}{RGB}{130,11,95}
\usepackage{multicol}
\widowpenalties 1 10000
\raggedbottom

\ifxetex
	\usepackage{fontspec} 
	\setmainfont{Roboto Slab}
	\setsansfont{Roboto}
	\setmonofont{Roboto Mono}
	\newfontfamily\condensed{Roboto Condensed}
	\newfontfamily\condensedlight{Roboto Condensed Light}
	\newfontfamily\light{Roboto Slab Light}
	\titleformat*{\section}{\LARGE\condensed\color{newCeten}}
	\titleformat*{\subsection}{\Large\condensedlight\color{newCeten}}
	\titleformat*{\subsubsection}{\large\condensedlight\color{newCeten}}
\else
	\usepackage[utf8]{inputenc} 
	\usepackage[T1]{fontenc} 
\fi

\title{Statuts du Ceten}
\author{Philippe Becker\\
	Kilian Cuny\\
	Julien Déoux\\
	Yoni Lévy} 
\date\today

%%%%%%%%%%%%%%
%  Document  %
%%%%%%%%%%%%%%

\begin{document}

	\pagenumbering{roman}

	%---------------%
	% Page de garde %
	%---------------%
	
	\begin{titlepage}
		\begin{center}
			\includegraphics[width=\textwidth]{images/ceten.png}\par
			\vspace{3cm}
			{\Huge \light Statuts}\par
			\vfill
			{\large Cercle des Élèves de TELECOM Nancy}\par
			{\large \light Association déclarée}
			\vfill
			{\light \today}\par
		\end{center}
	\end{titlepage}

	%--------------------%
	% Table des matières %
	%--------------------%
	
	%\tableofcontents
	%\clearpage

	%----------%
	% Articles %
	%----------%

	\pagenumbering{arabic}

	\section{Constitution et dénomination}
		Il est fondé, depuis le 10 janvier 1991 entre les adhérents aux présents
		statuts, une association à but non lucrative régie par la loi du 1er Juillet
		1901 et le décret du 16 août 1901, ayant pour titre “Cercle des Élèves de
		TELECOM Nancy” et pour sigle “CETEN”.

	\section{Objet et Buts}
	\label{sec:objet}
		Cette association a pour objet d’aider les élèves adhérents, anciens comme
		nouveaux, de TELECOM Nancy. Pour cela, elle se donne comme principaux
		objectifs :
		\begin{itemize}
			\item L’aide à la gestion de la vie extrascolaire en collaboration avec
			    les autres associations d'élèves de TELECOM Nancy, en proposant des
			    services et activités à l’ensemble de ses membres adhérents
			\item De favoriser l’entraide et la solidarité entre ses membres
			\item L’insertion des nouveaux élèves en première année
			\item De défendre et représenter les intérêts de ses adhérents
		\end{itemize}

		Tout autre objectif non-spécifique permettant la réalisation des objectifs
		susmentionnés.

	\section{Siège social}
		Le siège social est fixé :
		\begin{center}
			TELECOM Nancy\\
			193, Avenue Paul Muller\\
			CS 90172\\
			54602 Villers-lès-Nancy, FRANCE
		\end{center}

		Le transfert du siège social pourra être effectué par décision de l’assemblée
		générale.

	\section{Les caractères}
	\label{sec:caracteres}
		L’association prône des valeurs apolitiques, asyndicales et
		aconfessionnelles.
		Elle est ouverte à tout élève de TELECOM Nancy, ainsi qu’à toute personne
		dont la demande d’adhésion
		a été acceptée par le Bureau Des Élèves, sans distinction de race, d'âge, de
		sexe, de nationalité,
		de religion, d’opinion politique, d’orientation sexuelle, de condition
		physique.

	\section{Durée de l’association}
		La durée de l’association est illimitée.

	\section{Composition de l’association}
		L’association se compose de :
		\begin{itemize}
			\item Membres de fait
			\item Membres bienfaiteurs
			\item Membres adhérents
			\item Membres ponctuels
			\item Membres extérieurs.
		\end{itemize}
	
		Toutes ces personnes sont des personnes physiques.

	\section{Les membres}
		\subsection{Admission et adhésion}
			Pour faire partie de l'association, il faut :
			\begin{itemize}
				\item Dans le cas des membres adhérents, être ou avoir été élève
					de TELECOM Nancy
				\item Adhérer aux présents statuts
				\item À l’exception des membres de fait, bienfaiteurs et ponctuels,
					s'acquitter de la cotisation dont le montant est fixé par le
					Bureau Des Élèves.
			\end{itemize}

			Le Bureau Des Élèves pourra refuser des adhésions, avec avis motivé aux
			intéressés.

		\subsection{Les catégories de membres}
			\subsubsection{Les membres de fait}
				Peut être membre de fait :
				\begin{itemize}
					\item Tout ancien membre adhérent
					\item Tout membre actif d’une association étudiante en rapport
				    	avec TELECOM Nancy
					\item Tout enseignant, intervenant ou personnel de TELECOM Nancy
				\end{itemize}

				qui en fait la demande sous réserve d’acceptation du Bureau Des
				Élèves.

				Ce titre est décerné pour l’année scolaire en cours (de la date
				d’adhésion jusqu’au 31 août suivant) par le Bureau Des Élèves à
				l’exception du directeur de TELECOM Nancy qui est membre de fait
				permanent. Les membres de fait ont droit de participation et de vote
				aux assemblées générales.

			\subsubsection{Les membres bienfaiteurs}
				Peut être membre bienfaiteur toute personne physique ayant fait un
				don manuel à l’association sous réserve d’acceptation du Bureau Des
				Élèves.

			\subsubsection{Les membres ponctuels}
				Peut être membre ponctuel toute personne physique souhaitant
				participer à une manifestation ponctuelle ayant payé une cotisation
				d’un montant défini par le Bureau Des Élèves pour ladite
				manifestation, sous réserve d’acceptation du Bureau Des Élèves. Il ou
				elle n’est membre que pendant la durée de ladite manifestation.

			\subsubsection{Les membres adhérents}
				Est membre adhérent, toute personne souhaitant profiter des services
				et activités proposés par l’association qui s’est acquittée de la
				cotisation fixée annuellement. Les membres adhérents ont droit de
				participation et de vote aux assemblées générales.

			\subsubsection{Les membres extérieurs}
				Peut être membre extérieur toute personne non étudiante ou
				alumni de TELECOM Nancy souhaitant rejoindre l’association qui
				s'est acquittée d’une cotisation annuelle réduite fixée par le
				Bureau Des Élèves, sous réserve d'acceptation du Bureau Des
				Élèves. Cet acquittement n’est valide que pour l’année scolaire
				en cours (de la date d’adhésion jusqu’au 31 août suivant).

				Les membres extérieurs peuvent uniquement prétendre aux tarifs CETEN
				lors des manifestations organisées par l’association ainsi qu’aux
				avantages liés aux partenariats conclus avec l’association. Ils n’ont
				pas accès aux autres services ou activités de l’association. Un
				membre extérieur n’a pas le droit de participation ni de vote aux
				assemblées générales.

		\subsection{Perte de qualité de membre}
		\label{ssec:perte_qualite_membre}
			La qualité de membre se perd par :
			\begin{itemize}
				\item Décès
				\item Radiation prononcée par le Bureau Des Élèves pour non-paiement
                    de la cotisation, l'intéressé ayant été invité à fournir des
                    explications devant ledit Bureau
				\item Exclusion prononcée par le Bureau Des Élèves pour infraction
			        aux présents statuts, au règlement intérieur de
					l'association, ou pour tout autre motif portant préjudice
					aux intérêts moraux et matériels de l’association,
					l'intéressé ayant été invité à fournir des explications
					devant ledit Bureau et par écrit
				\item Démission adressée par écrit au président de l’association
			\end{itemize}

			Un membre ayant été radié ne pourra prétendre à une nouvelle adhésion
			qu’après avis favorable du bureau et en tout état de cause, pas avant
			l’année scolaire suivant sa radiation.

	\section{Affiliation}
		La présente association peut adhérer à d’autres associations, unions ou
		regroupements par décision du Bureau Des Élèves tant que cette affiliation
		n’entre pas en conflit avec le caractère et les objectifs de l’association
		décrits dans les articles~\ref{sec:objet} et~\ref{sec:caracteres} des
		présents statuts.

	\section{Ressources et Financement}
		\subsection{Ressources}
			Les ressources de l’association comprennent :
			\begin{itemize}
				\item Le montant des cotisations
				\item Les dons manuels
				\item Les subventions diverses
				\item Les manifestations
				\item Toutes les ressources autorisées par les lois et le règlement
			        en vigueur
			\end{itemize}

		\subsection{Exercice comptable}
			L’exercice comptable commence au 1\up{er} janvier et finit le 31
			décembre de chaque année. Tout membre de l’association peut, sur simple
			demande au trésorier et en présence de celui-ci, accéder aux comptes de
			l’association.

	\section{Organes souverains}
		L’association est administrée par :
		\begin{itemize}
			\item L’Assemblée Générale
			\item Le Bureau Des Élèves
		\end{itemize}

		\subsection{Dispositions Communes aux deux Assemblées Générales}
			\subsubsection{Composition des Assemblées Générales}
				Les assemblées générales, ordinaire ou extraordinaire, peuvent
				comprendre tous les membres de l’association ayant droit de
				participation à ladite assemblée générale.

			\subsubsection{Convocation et ordre du jour}
			\label{sssec:convocation}
				Quinze jours avant la date fixée par le Bureau Des Élèves, les
				membres de l’association sont convoqués par les soins du secrétaire
				de l’association. La convocation se fait par courrier électronique. 
				L’ordre du jour fixé et prévu par le Bureau Des Élèves, sera envoyé
				aux membres en même temps que la convocation et seront modifiables
				jusqu’à sept jours avant l'assemblée générale. Seuls les points
				inscrits à l’ordre du jour pourront être abordés.
				
				Tout membre de l'association peut, s'il le souhaite, soumettre
				un point à ajouter à l'ordre du jour au Bureau Des Élèves sous
				condition que cet ajout soit constructif et non contraire au but
				et au caractère de l'association. Ce membre, ou le membre qui le
				représente, aura l'obligation dans le cas ou son point est
				ajouté par validation du Bureau Des Élèves, de faire entendre à
				l'assemblée générale son avis sur ledit point sans quoi celui-ci
				ne sera pas abordé.

			\subsubsection{Président et Secrétaire de l’Assemblée Générale}
				La présidence de l’assemblée générale appartient au président de
				l’association ou à un membre du Bureau Des Élèves choisi par ledit
				président si celui-ci est empêché.

				La rédaction du procès verbal de l’assemblée générale appartient au
				secrétaire de l’association ou à un membre du Bureau Des Élèves s’il
				est empêché. Les délibérations sont constatées par des procès-verbaux
				inscrits sur un registre et signés par le président et le secrétaire.

			\subsubsection{Délibérations}
				Les délibérations se font à main levée sauf demande de la part d’un
				membre adhérent. Les décisions sont prises à la majorité des
				suffrages exprimés, c’est-à-dire des membres présents ou représentés.
				En cas de partage, la voix du président est prépondérante. 

			\subsubsection{Représentation des membres en cas d’empêchement}
				Les membres de l’association peuvent se faire représenter par un
				autre membre de l’association en cas d’empêchement. Un membre présent
				ne peut détenir plus d’un mandat de représentation.

				Il est tenu une feuille de présence signée par chaque membre présent
				et certifiée par le président de l’assemblée.

		\subsection{L’Assemblée Générale Ordinaire (AGO)}
			Elle se réunit chaque année au mois de Janvier. Elle peut délibérer
			valablement quel que soit le nombre de membres présents ou représentés.
			Le président de l’assemblée générale fait entendre à l’assemblée générale
			la situation morale, l’activité de l’association. Le trésorier de
			l’association rend compte à l’assemblée générale de la gestion
			financière. Après avoir délibéré et statué les divers rapports,
			l’assemblée générale délibère sur le budget prévisionnel ainsi que sur
			les autres points à l’ordre du jour. 

		\subsection{L’Assemblée Générale Extraordinaire (AGE)}
			Elle peut se réunir à la demande du Bureau Des Élèves ou à la demande de
			la moitié plus un des membres de l’association, sur convocation du
			président de l’association, suivant les modalités prévues dans
			l’article~\ref{sssec:convocation}.

			\subsubsection{Quorum}
				Dans le cas de l’Assemblée Générale Extraordinaire, les délibérations
				ne peuvent être validées que si la majorité plus un
				des membres de l’association est présente ou représentée. Si le
				quorum n’est pas atteint, une nouvelle Assemblée Générale
				Extraordinaire
				est convoquée dans un délai d’au moins 7 jours, avec le même ordre du
				jour mais sans nécessité de quorum.

				Un membre représentant le quorum de l’assemblée générale peut, s’il
				le souhaite, soumettre au président de l’association un point à
				ajouter à l’ordre du jour. Ce point sera ajouté s’il a été
				soumis au moins 3 jours avant l’assemblée générale
				extraordinaire, sous condition que cet ajout soit constructif et
				non contraire à l'objet et au caractère de l’association.

			\subsubsection{Modification des Statuts}
				Les statuts de l’association peuvent être modifiés à la demande de la
				majorité du Bureau Des Élèves. Tout membre peut, s’il le souhaite,
				soumettre une idée de modification des statuts au Bureau Des Élèves
				lors des réunions dudit bureau, sous condition que cette modification
				soit constructive et non contraire au but et au caractère de
				l’association, sauf si elle concerne l'article~\ref{sec:objet}
				ou l'article~\ref{sec:caracteres}.

				Toute modification de statut doit être rédigée et envoyée aux membres
				de l’association au plus tard en même temps que la convocation pour
				l’Assemblée Générale Extraordinaire durant laquelle seront votés les
				changements de statuts.

			\subsubsection{Dissolution}
				La dissolution de l’association ne peut être prononcée que par
				l’assemblée générale convoquée spécialement à cet effet.
				En cas de dissolution de l’association, l’actif net subsistant et les
				documents comptables seront remis obligatoirement à une ou plusieurs
				associations poursuivant des buts similaires et qui seront désignées
				par l’assemblée générale extraordinaire. Les comptes afférents seront
				clos.
				La dissolution doit faire l’objet d’une déclaration à la préfecture
				ou à la sous-préfecture du siège social.

		\subsection{Le Bureau Des Élèves}
		\label{ssec:bde}
			\subsubsection{Fonction}
				Le Bureau Des Élèves (BDE) est l'organe exécutif de l'association, il
				est placé sous la direction du président. Il assure la gestion de
				l'association entre deux assemblées générales dans le but de mettre
				en oeuvre les décisions de la dernière assemblée générale et
				conformément à l'objet des statuts.

			\subsubsection{Composition}
				Le Bureau Des Élèves est composé de 10 élèves de TELECOM Nancy élus
				pour un mandat d’un an, débutant le 1\up{er} janvier et finissant le
				31 décembre. Dans les deux semaines suivant son élection, le nouveau
				Bureau des Élèves doit se réunir afin d'attribuer les postes suivants
				:
				\begin{itemize}
					\item Un Président, élève de deuxième année
					\item Un Vice-président, élève de première année
					\item Un Trésorier assisté d’un Vice-trésorier
					\item Un Secrétaire
					\item Un Responsable des clubs
					\item Un Responsable Informatique
					\item Un Responsable Logistique
					\item Deux Responsables Communication.
				\end{itemize}

				Le cumul des postes énumérés ci-dessus est interdit.

				En plus de ces 10 élèves :
				\begin{itemize}
					\item est choisi en la personne du président sortant un
						représentant des troisième année. Dans le cas où le président
						sortant n’est pas en troisième année à TELECOM Nancy au
						moment des élections ou est dans l'incapacité d'assurer
						cette fonction, un autre membre du bureau sera alors
						choisi par le Bureau des Élèves sortant. Dans le cas
						extrême où aucun membre du bureau sortant n’est en
						troisième année, le bureau sortant choisira un membre
						adhérent de l'association, élève de troisième année.
					\item est élu par l’ensemble des membres adhérents en filière
						apprentissage un responsable apprentis.
				\end{itemize}

			\subsubsection{Délibérations}
				Les délibérations se font à main levée sauf demande de la part d’un
				membre du Bureau Des Élèves. Les décisions sont prises à la majorité
				des	suffrages exprimés c’est-à-dire des membres présents ou
				représentés. En cas de partage, la voix du président est
				prépondérante.

				L’ensemble de ses pouvoirs doit être soumis à la validation de la
				majorité absolue du Bureau Des Élèves. Les membres du Bureau Des
				Élèves peuvent se faire représenter par un autre membre du Bureau en
				cas d’empêchement. Un membre présent ne peut détenir plus d’un mandat
				de représentation.

			\subsubsection{Pouvoirs}
				Le Bureau Des Élèves :
				\begin{itemize}
					\item Peut autoriser tout acte ou opération qui n’est pas
					    statutairement de la compétence de l’assemblée générale
					    ordinaire ou extraordinaire.
					\item Se prononce sur les admissions de membres de l’association
					    et confère les éventuels titres de membres de fait et
    					bienfaiteurs.
					\item Se prononce également sur les mesures de radiation et
	    				d’exclusion des membres.
					\item Contrôle la gestion des clubs qui doivent lui rendre compte
		    			de leur activité à l’occasion de ses réunions.
					\item Autorise l’ouverture de tous comptes bancaires ou postaux,
					    effectue tout emploi de fonds, contracte tout emprunt
					    hypothécaire ou autre, sollicite toute subvention, requiert
					    toutes inscriptions ou transcriptions utiles.
					\item Autorise le président ou le trésorier à exécuter tout acte,
					    aliénation ou investissement reconnu nécessaire, des biens et
					    des valeurs appartenant à l’association et à passer les
					    marchés et contrats nécessaires à la poursuite de son objet.
				\end{itemize}

			\subsubsection{Modalité des élections du Bureau Des Élèves}
			\label{sssec:elections}
				L’élection se fait exclusivement à bulletin secret. Le vote par
				procuration est admis dans la limite d’une procuration par personne.
				Elle doit être datée et signée, rédigée sur les formulaires fournis
				par le Bureau Des Élèves. Le vote par correspondance est interdit.
				Dans le cas où l’électeur est porteur d’une procuration, il doit
				introduire autant de bulletins que le nombre de voix qu’il représente
				et doit émarger devant le nom de la personne dont il porte la
				procuration. Il doit remettre ses procurations au responsable du
				bureau de vote.
				
				Dans le cas où il y a une ou deux listes, seul un tour est
				nécessaire. Pour que les bulletins soient déclarés recevables, on
				effectue alors un scrutin plurinominal majoritaire à un tour avec
				panachage dont le détail du fonctionnement est le suivant :

				Sont élus les cinq premiers candidats de première année et les cinq
				premiers candidats de deuxième année ayant eu le plus de voix à
				condition que le nombre total de votants soit supérieur à 33\% des
				électeurs inscrits.

				Dans le cas où le nombre de votants n’est pas supérieur à 33\% des
				électeurs inscrits, le vote est prolongé d’un jour ouvré. Si à la fin
				de ce jour, le nombre de votants n’est toujours pas supérieur à 33\%
				des électeurs inscrits, le dépouillement a quand même lieu.

				Dans le cas où il y a plus de deux listes, pour que les bulletins
				soient déclarés recevables, on effectue alors un scrutin plurinominal
				majoritaire à deux tours avec panachage dont le détail du
				fonctionnement est le suivant :

				Sont élus au premier tour dans l’ordre décroissant des voix les
				candidats ayant eu une majorité absolue des voix à condition que le
				nombre de voix recueillies soit supérieur à 33\% des électeurs
				inscrits. Les votes pour et blanc comptent dans le nombre total de
				voix exprimées. Les votes nuls ne comptent pas dans le nombre total
				de voix exprimées.

				Si cinq étudiants de première année et cinq de deuxième année sont
				élus, les élections s’achèvent.

				Dans le cas contraire, un second tour est organisé dans un délai de
				cinq jours ouvrables. Par année d’étude, seuls peuvent se maintenir
				les candidats non élus ayant obtenu les voix d’au moins 25\% des
				inscrits ou, à défaut, les six candidats non élus ayant obtenu le
				plus de voix. En cas d’égalité du nombre de voix au premier tour et
				si cette égalité empêche de choisir les candidats admissibles au
				second tour, ces candidats sont admis au second tour.

				Au second tour, en cas d'égalité de suffrages entre plusieurs élèves
				d'une même année et si cette égalité empêche de désigner les cinq
				élèves entrant au Bureau des Élèves, un nouveau tour portant sur ces
				seuls élèves sera organisé dans un délai de cinq jours ouvrables afin
				de les départager, et ainsi de suite jusqu’à départage entre ces
				candidats.

				Dans tous les cas, tout bulletin incomplet, corrigé, raturé ou
				comprenant une marque distinctive sera déclaré nul. Tout bulletin
				sans aucune écriture sera considéré comme blanc.

			\subsubsection{Vacances}
				En cas de vacance d'un poste, le Bureau Des Élèves pourvoit
				provisoirement au remplacement du membre. Il sera procédé à son
				remplacement définitif lors de la prochaine assemblée générale. Les
				pouvoirs des membres ainsi élus prennent fin à l’époque où devrait
				normalement expirer le mandat des membres remplacés. Si plus d’un
				tiers des membres du Bureau Des Élèves devaient laisser leur poste
				vacant, la nouvelle composition du bureau devra être validée par une
				Assemblée Générale Extraordinaire.

				La vacance du poste de Président ou de Trésorier entraînera
				obligatoirement la convocation d’une Assemblée Générale
				Extraordinaire dans un délai d’un mois afin de prendre les mesures
				qui s’imposent.

	\section{Règlement intérieur}
		Un règlement intérieur sera déposé par le Bureau Des Élèves et adopté par
		ledit Bureau à la majorité absolue afin de fixer les divers points non prévus
		dans les statuts, notamment ceux qui ont trait au descriptif des postes du
		Bureau, au mode d’utilisation des locaux et des divers équipements, à
		l’administration interne de l’association et aux montants des cotisations. Il
		ne pourra comprendre aucune disposition contraire aux statuts.

		Ce règlement intérieur peut être modifié par le Bureau Des Élèves. Il est mis
		à disposition à l’ensemble des membres ainsi qu’à chaque nouvel adhérent. Le
		nouveau règlement intérieur sera adressé à chacun des membres de
		l'association par affichage et courrier électronique sous un délai de 7 jours
		suivant la date de la modification. Le délai d’application est de 7 jours
		après la diffusion du règlement intérieur.

		Les articles concernant l’utilisation des locaux seront déposés par le Bureau
		Des Élèves et adoptés en accord avec le directeur de TELECOM Nancy.	Leur
		modification nécessite l’accord du directeur de TELECOM Nancy.

	\vspace*{5cm}
	\begin{center}
		{\large\light Les présents statuts ont été votés lors de l’Assemblée Générale
		Extraordinaire à Villers-lès-Nancy le 17 octobre 2016.}
	\end{center}
    \vspace{3cm}
	Signatures\par
	\begin{multicols}{3}
	    \begin{center}
	        Président \\
	        Trésorier \\
	        Secrétaire
	    \end{center}
	\end{multicols}
    
\end{document}
